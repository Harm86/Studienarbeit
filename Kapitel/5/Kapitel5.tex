\chapter{Fazit}
%
%\section{Problemstellungen}
%Bsp: Mesh, Sensoren falsch verlötet
\section{Zusammenfassung}
	Das Studienarbeitsthema war insgesamt interessant und lehrreich. Die in den Vorlesungen vermittelten Inhalte konnten großteils angewandt werden. Die Umsetztungszeit war trotz Hürden, wie Klausuren und der dazwischen liegenden Praxisphase ausreichend gestaltet. Während der Orientierungsphase haben wir viel Zeit mit dem Thema Mesh verbracht, welches leider nicht umgesetzt werden konnte. Trotz vieler Anläufe und verschiedener Ansätze ist die Umsetzung eines Mesh Netzes wegen den gegebenen Wlanadaptern gescheitert. Die Adapter waren notwendig, da weder der Raspberry Pi Modell 1 und Modell 2 einen eingebautes Wlanmodul besitzen. Mangels mehrer Raspberry Pis des Modells 3 konnten wir ein Mesh Netz aus diesen nicht testen. 
	
	Des Weiteren war die Einarbeitungsphase sehr Zeitintensiv. Gründe dafür war die geringe Erfahrung mit den Raspberry Pi und die fehlende Erfahrung mit den Sensoren. Der Umgang mit den Unixsystem ist allen zwar geläufig, doch die Eigenheiten der Raspberry Pi Umgebung sind erst durch "'Trial and Error"' ans Tageslicht gekommen. Eines der Probleme war beispielsweise die schlechte Kompatibilität der ausgewählten Raspberry Pi Distribution mit dem lokalen Wlan Netz. Weiterhin haben wir uns bei der allgemeinen Zeitplanung verschätzt. 
	
	Da die Studienarbeit zu Beginn des fünften Theoriesemesters begann haben wir den Vorbereitungsaufwand der jeweiligen Klausuren schlecht eingeschätzt. Wir mussten früher als erwartet die Arbeit am Projekt eindämmen und mit der Klausurvorbereitung weitermachen. 
	
	Die Vorlesungen hatten einen positiven Effekt auf unsere Projektarbeit. So wurde Beispielsweise in einer Vorlesung die theoretischen Grundlagen für die Schnittstellenprogrammierung näher gebracht. Dennoch war die praktische Umsetzung, aufgrund der Beschaffenheit von den verwendeten Programmiersprachen, zeitintensiv. Zusätzlich wurde die Programmierqualität durch die Vorlesungen positiv beeinflusst.
	
	Ein weiterer Zeitfaktor war die Beschaffung der Sensoren. Diese mussten sorgfältig gewählt werden und hatten eine relativ lange Lieferzeit.
	Die Sensoren noch eine weitere Hürde. Erst nach einer sehr langen Fehlersuche ist uns aufgefallen, dass einige Sensoren defekt geliefert wurden. Diese mussten dann durch das Umlöten korrigiert werden. Dieser Vorgang war ebenfalls mit einem Risiko behaftet. Das Löten erforderte präzises Vorgehen, damit die Sensoren funktinal bleiben. Wäre uns dabei ein Fehler unterlaufen, dann müssten neue Ersatzsensoren beschafft werden, die ebenfalls eine lange Lieferzeit haben. Dank sorgfältigem Umgang konnte aber dieses Problem gelöst werden.
	
	Im Großen und Ganzen das Projekt dennoch Erfolgreich. Die vorgegebenen Ziele wurden erreicht - mit der Ausnahme vom Mesh Netz. Wir konnten dafür weitere Funktionen, wie die Unterstützung einer Webcam und die Statistikseite, umsetzten. Die Entwicklung mit der Skriptsprache Python war am Anfang, aufgrund der Eigenheiten von Python, etwas umständlich. Doch nach einer kurzen Gewöhnungszeit wurde die Sprache intuitiv und für unsere Ziele nützlich.
	
	Unsere Erfahrungen in den Bereichen Netzwerktechnik und Datenbankadministration waren ebenfalls prositive Faktoren für das Projekt.
	Aufgrund der Arbeitserfahrung der Netzwerkvorlesungen war die Gestaltung und Umsetzung des notwendigen Netzwerkes für die Raspberry Pis kein Problem.  
	Ebenso war die Administration der Datenbank und die Implementierung der Connectoren zu dieser dank der praktischen Erfahrung gut umsetzbar.
	
	Die Umsetzung der Studienarbeit in einer Gruppenarbeit hatte ihre Vor- und Nachteile. Zu den Vorteilen einer Gruppenarbeit im Vergleich zu einer Einzelarbeit ist zu einem das größere Wissensspektrum, die Diskussionsmöglichkeit und die Motivation. Jedes Gruppenmitglied hat seine Kenntnisse, die in dem Projekt genutzt werden können. Die Möglichkeit einer Diskussion bietet eine gute Grundlage für das Lösen von Problemstellungen. Auf der anderen Seite hatte die Gruppenarbeit auch ihre Nachteile für die Studienarbeit. Abstimmungen zwischen den Gruppenmitgliedern müssen regelmäßig gemacht werden, damit keiner versehentlich gegen das Projekt arbeitet. Des Weiteren treffen hierbei verschiedene Schreibstile aufeinander. Sei es der Programmierstil oder der Schreibstil, so mussten beide für das Projekt entsprechend angepasst werden. Zusätzlich sind Diskussionen oftmals sehr zeitintensiv.
	
	Beim Implementieren der Webseite ist uns aufgefallen, dass unser eigener Ansatz eines sicheren Logins nicht unseren Anforderungen genügt. Somit haben wir beschlossen ein entsprechendes Framework von einer anderen Quelle zu beziehen, da diese wesentlich ausgereifter ist als unsere Umsetzung. 
	
	Die Umsetzung der Studienarbeit bietet für unerfahrene Bastler, die gerne so ein System zuhause nachbauen wollen, eine benutzerfreundliche Möglichkeit dies umzusetzen. Die große Raspberry Pi Community bietet ebenfalls Hilfestellungen bei Problemen und Verbesserungsideen.
	
%	- Zeit war ausreichen
%		- Arber zu viel Zeit in Mesh bei der Orientierungsphase verbracht
%	- Einarbeitsphase sehr Zeitintensiv
%		- haben uns da etwas verschätzt
%		- Viele Module im 5. Semester -> wenig zeit für die Studienarbeit
%		- Keine Erfahrung bei Sensoren
%		- Wenig Erfahrung mit dem Thema Raspberry Pi
%	- Die Theorie der Schnittstellen wurde in der Vorlesung beigebracht
%		- Eigene Umsetzung hat dennoch Zeit gekostet
%	- Baschaffung der Sensoren (Lieferzeit)
%	- Defekte Sensoren 
%		- Umlöten der Sensoren
%	- Alle gesetzten Ziele außer Mesh erreicht
%		- Zusätzlich andere Ziele : Webcam, Statistik
%	- Umsetzung mit Python relativ umständlich
%		- Python war überraschend intuitiv
%	- Aufbau des Netzwerkes war einfach (aufgrund Kenntnisse: Arbeit und Vorlesung)
%	- Datenbankadministration und Connector dank Arbeitserfahrung gut machbar
%	- Gruppenarbeit hat Vor und Nachteile
%		- Vorteile:
%			- Viel Wissen
%			- Disskussion führt zu guten ideen + Problemlösung
%			- Gegenseitige Motivation
%		- Nachteile:
%			- Großer Abstimmungsbedarf
%				- Programmierstile treffen aufeinander
%				- Insbesondere Doku -> Verschiedene Schreibstile
%			- Disskussionen sind Zeitintensiv
%				- Zu viele Ideen (teilweise)
%	- Login von anderer Quelle gewählt
%		- Eigene Entwicklung nicht ausreichend für die Vorgaben
%		- In hinblick auf Sicherheit deutlich ausgereifter als eigene Entwicklung
%	- Abschluss
%		- Benutzerfreundlich
%		- Leicht zuhause nachbaubar
%		- Große Community hilft das thema weiterleben zulassen
\section{Ausblick}
	
	Wie zuvor angemerkt hat dieses Projekt noch Verbesserungspotential. So kann das Mesh Netz in zukünftigen Ansätzen umgesetzt werden. Eine eigene automatisierte Email Benachrichtigung ist ein "'nice to have"' Feature, dass umgesetzt werden könnte. Um das Projekt für nicht Bastler noch benutzerfreundlicher zu machen, kann ein automatisiertes Provisioning für eine Destributionsverteilung umgesetzt werden. Die in diesem Ansatz genutzten Sensoren sind nur ein kleiner Teil der möglichen Sensoren. So könnte zukünftig andere Sensoren, wie beispielsweise ein Gassensor, umgesetzt werden. Ferner könnte die Möglichkeit der Konfiguration der Zentrale und der Sensorknoten mittels der Webseite die Anwendung bequemer für den Nutzer machen. In dieser Konfigurationsfunktion könnte der Nutzer explizite Cronjobs über die Webseite erstellen. Zum Beispiel könnte ein täglicher Cronjob einen Report an den Nutzer senden. Der Report könnte den technischen Zustand der Zentrale beschreiben oder den zeitlichen Verlauf der Messdaten visualisieren.
	
	Statt dem verwendeten Zeitgeber könnte ebenfalls ein eigener \ac{NTP} umgesetzt werden. Neben dem Zeitserver würde ein eigener \ac{DNS} das ansprechen der jeweiligen Sensorknoten oder der Zentraleinheit vereinfachen.
	
	Abschließend wäre das nächste Ziel eine eigene Benutzerverwaltung mittels der Webseite. Hier könnte der Administrator in einer eigenen Ansicht neue Nutzer mit bestimmten Rechten anlegen.
	%- Mesh
	%- Automatisierte Email Notification
	%- Automatisches provisioning (Imageverteilung)
	%- Weitere Sensoren
%		- Andere A/D Wandler
%	- Konfiguration der Zentrale und Sensorknoten über Webseite
%		- explizite Crons (Daily, Weekly) über Webseite erstellen
%			- Cron: Reports erstellen
%	- NTP selbst machen (nicht den vom JOY-IT)
%	- DNS (über Namen ansprechen)
%	- Benutzerverwaltung mit unterschiedlichen Rechten
%		- Adminansicht