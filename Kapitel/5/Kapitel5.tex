\chapter{Fazit}
%
%\section{Problemstellungen}
%Bsp: Mesh, Sensoren falsch verlötet
\section{Zusammenfassung}
	Das Studienarbeitsthema war insgesamt interessant und lehrreich. Die in den Vorlesungen vermittelten Inhalte konnten großteils angewandt werden.
	- Zeit war ausreichen
		- Arber zu viel Zeit in Mesh bei der Orientierungsphase verbracht
	- Einarbeitsphase sehr Zeitintensiv
		- haben uns da etwas verschätzt
		- Viele Module im 5. Semester -> wenig zeit für die Studienarbeit
		- Keine Erfahrung bei Sensoren
		- Wenig Erfahrung mit dem Thema Raspberry Pi
	- Die Theorie der Schnittstellen wurde in der Vorlesung beigebracht
		- Eigene Umsetzung hat dennoch Zeit gekostet
	- Baschaffung der Sensoren (Lieferzeit)
	- Defekte Sensoren 
		- Umlöten der Sensoren
	- Alle gesetzten Ziele außer Mesh erreicht
		- Zusätzlich andere Ziele : Webcam, Statistik
	- Umsetzung mit Python relativ umständlich
		- Python war überraschend intuitiv
	- Aufbau des Netzwerkes war einfach (aufgrund Kenntnisse: Arbeit und Vorlesung)
	- Datenbankadministration und Connector dank Arbeitserfahrung gut machbar
	- Gruppenarbeit hat Vor und Nachteile
		- Vorteile:
			- Viel Wissen
			- Disskussion führt zu guten ideen + Problemlösung
			- Gegenseitige Motivation
		- Nachteile:
			- Großer Abstimmungsbedarf
				- Programmierstile treffen aufeinander
				- Insbesondere Doku -> Verschiedene Schreibstile
			- Disskussionen sind Zeitintensiv
				- Zu viele Ideen (teilweise)
	- Login von anderer Quelle gewählt
		- Eigene Entwicklung nicht ausreichend für die Vorgaben
		- In hinblick auf Sicherheit deutlich ausgereifter als eigene Entwicklung
	- Abschluss
		- Benutzerfreundlich
		- Leicht zuhause nachbaubar
		- Große Community hilft das thema weiterleben zulassen
\section{Ausblick}
	- Mesh
	- Automatisierte Email Notification
	- Automatisches provisioning (Imageverteilung)
	- Weitere Sensoren
		- Andere A/D Wandler
	- Konfiguration der Zentrale und Sensorknoten über Webseite
		- explizite Crons (Daily, Weekly) über Webseite erstellen
			- Cron: Reports erstellen
	- NTP selbst machen (nicht den vom JOY-IT)
	- DNS (über Namen ansprechen)
	- Benutzerverwaltung mit unterschiedlichen Rechten
		- Adminansicht