\chapter{Umsetzung}

\section{Allgemein}

\subsection{DHCP}

Als DHCP-Server wurde der ISC-DHCP-Server verwendet.\\
Die IP-Adressen werden nur über das wlan0-Interface der Zentraleinheit vergeben.
Als Netz wurde das Private Netz 192.168.178.0 /24 verwendet. In diesem Netz hat
Die Zentraleinheit als DHCP-Server die Adresse 192.168.178.1 /24. Diese Adresse
ist Statisch eingetragen. Alle anderen Geräte erhalten dynamische IP-Adressen
aus dem Bereich 192.168.178.10 - 192.168.178.250. Die Lease-Time wurde auf
604800 sekunden festgelegt. Dies entspricht 7 Tagen. Da nur wenige Geräte im
Netz verfügbar sind und auch keine häufigen Änderungen erwartet werden wird dies
als ausreichend angesehen.

\begin{verbatim}
/etc/dhcp/dhcpd.conf

#Rogue-DHCP-Server nicht erlauben (Doppelter DHCP-Server)
authoritative;

#Definition des Subnetzes
subnet 192.168.178.0 netmask 255.255.255.0
{
        #Angabe der DHCP-Range
        range 192.168.178.10 192.168.178.250;

        #Angabe der Lease-Times 7 Tage in sekunden
        default-lease-time 604800;
        max-lease-time 604800;

        #Begrenzung auf das WLAN-Interface
        interface wlan0;
}

\end{verbatim}

\subsection{WLAN}

Das WLAN ist mit dem Namen Pinet konfiguriert.


\begin{verbatim}
/etc/hostapd/hostapd.conf

interface=wlan0
#driver=rtl871xdrv
ssid=Pinet
country_code=DE
hw_mode=g
channel=6
macaddr_acl=0
auth_algs=1
ignore_broadcast_ssid=0
wpa=2
wpa_passphrase=IrgendeinbloedesPasswort
wpa_key_mgmt=WPA-PSK
wpa_pairwise=CCMP
wpa_group_rekey=86400
ieee80211n=1
wme_enabled=1

\end{verbatim}

\subsection{Mesh}

\subsection{Verschlüsselung ?}

\section{Sensoren}

\subsection{Sensoreinrichtung}

\subsection{Programmcode}

\section{Namenskonvention}
