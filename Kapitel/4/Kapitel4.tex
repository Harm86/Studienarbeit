\chapter{Umsetzung}

\section{Allgemein}

\subsection{DHCP}

Als DHCP-Server wurde der ISC-DHCP-Server verwendet.\\
Die IP-Adressen werden nur über das wlan0-Interface der Zentraleinheit vergeben.
Als Netz wurde das Private Netz 192.168.178.0 /24 verwendet. In diesem Netz hat
Die Zentraleinheit als DHCP-Server die Adresse 192.168.178.1 /24. Diese Adresse
ist Statisch eingetragen. Alle anderen Geräte erhalten dynamische IP-Adressen
aus dem Bereich 192.168.178.10 - 192.168.178.250. Die Lease-Time wurde auf
604800 sekunden festgelegt. Dies entspricht 7 Tagen. Da nur wenige Geräte im
Netz verfügbar sind und auch keine häufigen Änderungen erwartet werden wird dies
als ausreichend angesehen.

\begin{verbatim}
/etc/dhcp/dhcpd.conf

#Rogue-DHCP-Server nicht erlauben (Doppelter DHCP-Server)
authoritative;

#Definition des Subnetzes
subnet 192.168.178.0 netmask 255.255.255.0
{
        #Angabe der DHCP-Range
        range 192.168.178.10 192.168.178.250;

        #Angabe der Lease-Times 7 Tage in sekunden
        default-lease-time 604800;
        max-lease-time 604800;

        #Begrenzung auf das WLAN-Interface
        interface wlan0;
}

\end{verbatim}


\subsection{WLAN}

Es wird ein Funknetz auf Basis des 802.11n Standards verwendet. Als Name wurde
Pinet festgelegt, der von allen gesehen werden kann. In der Tabelle
(\nameref{tab:WLAN-Konfiguration}) sind die einzelnen Optionen
aufgeführt und erläutert.


\begin{table}
\caption{WLan-Konfigurationsdetails}
\label{tab:WLAN-Konfiguration}
\begin{tabular}{p{0.5\textwidth} p{0.45\textwidth}}
Befehl & Erklärung \\
interface=wlan0 & Das Interface auf dem das Funknetz ausgestrahlt wird \\
ssid=Pinet & Der name des Funknetzes \\
country\_code=DE & Über die Festlegung der Region wird sichergestellt, dass das
Funknetz die spezifischen Grenzwerte für Kanäle oder Sendestärke einhält \\
hw\_mode=g & legt fest, dass das Funknetz im 2,4 GHz-Band ausgestrahlt wird \\
channel=6 & Der Funkkanal 6 wird verwendet \\
macaddr\_acl=0 & MAC-Adressenfilterung ist deaktiviert \\
auth\_algs=1 & Legt fest, dass als Verschlüsselung WPA verwendet wird \\
ignore\_broadcast\_ssid=0 & Die SSID wird ausgestrahlt und nicht versteckt. \\
wpa=2 & Legt die WPA-Version fest auf WPA2 \\
wpa\_passphrase=IrgendeinbloedesPasswort & Legt den Pre-Shared-Key fest \\
wpa\_key\_mgmt=WPA-PSK & Legt fest, dass ein Pre-Shared-Key verwendet wird \\
wpa\_pairwise=CCMP & Legt fest, dass nur der AES-Verschlüsselungsalgorithmus
verwendet wird \\
wpa\_group\_rekey=86400 & Legt fest, dass alle 86400 Sekunden ein neuer
Schlüssel verwendet werden muss \\
ieee80211n=1 & Aktiviert den n-Standard \\
wme\_enabled=1 & Aktiviert Quality-of-Service - Voraussetzung für die Verwendung
des n-Standards \\
 \end{tabular}
\end{table}

\subsection{Verschlüsselung}

Das WLAN ist mit WPA2 verschlüsselt. WPA2 gilt aktuell als sicher, was nicht für
die Alternativen WEP oder WPA gilt. Zur Authentifizierung wird ein
Pre-Shared-Key (PSK) verwendet.

\subsection{Mesh}

\subsection{Website}

\subsubsection{Login}

\subsubsection{Übersicht}

\subsubsection{Statistik}

\subsubsection{Webcam}

\subsubsection{Impressum}


\section{Sensoren}

\subsection{Sensoreinrichtung}

\subsection{Programmcode}

\section{Namenskonvention}
