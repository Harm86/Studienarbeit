\chapter{Grundlagen}

\section{Programmiersprachen}

\subsection{Python vs. Java}

\subsection{MySQL}

\subsection{PHP}

\section{Raspberry Pi}
Der Raspberry Pi wurde von der britischen Raspberry Pi Foundation entworfen um
jungen Menschen den Erwerb von Programmier- und Hardwarekenntnissen zu
ermöglichen. Er ist ein Einplatinencomputer und für wenig Geld verfügbar. Der
Raspberry Pi zeichnet sich durch frei programmierbare Schnittstellen aus um
beispielsweise Sensoren anzuschließen.

 Mittlerweile gibt es mehrere Modelle:

\begin{itemize} 
\item Pi Zero 
\item Pi Zero W
\item Pi 1 Modell A
\item Pi 1 Modell A+
\item Pi 1 Modell B
\item Pi 1 Modell B+
\item Pi 2 Modell B
\item Pi 3 Modell B 
\end{itemize}


\section{Konfiguration}

Die Sensoren werden an einen Raspberry Pi angeschlossen und melden die
gemessenen Werte an einen zentralen Raspberry Pi. Der zentrale Raspberry Pi legt
die gemeldeten Daten in einer Datenbank ab. Eine Website greift auf die
Datenbank zu und stellt die Daten dar.\\
Die Kommunikation wird über ein eigenes WLAN-Netz abgewickelt das von der
Zentraleinheit aufgespannt wird. IP-Adressen werden von einem DHCP-Server, der
auf der Zentraleinheit installiert ist vergeben.\\
Die Website wird durch einen Apache-Webserver auf der Zentraleinheit
bereitgestellt.

\section{Sensoren}

Es werden folgende Sensoren verwendet:

\begin{itemize}
\item Feuersensor
\item
\end{itemize}

\section{Vernetzung}

Die Vernetzung erfolgt über ein Funknetz nach dem WLAN-Standard 802.11n. 

\section{Verschlüsselung}

Das WLAN ist mit WPA2 verschlüsselt. WPA2 gilt aktuell als sicher, was nicht für
die Alternativen WEP oder WPA gilt. Zur Authentifizierung wird ein
Pre-Shared-Key (PSK) verwendet.
