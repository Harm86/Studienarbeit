\chapter{Grundlagen}

\section{Raspberry Pi}
Der Raspberry Pi wurde von der britischen Raspberry Pi Foundation entworfen um
jungen Menschen den Erwerb von Programmier- und Hardwarekenntnissen zu
ermöglichen. Er ist ein Einplatinencomputer und für wenig Geld verfügbar. Der
Raspberry Pi zeichnet sich durch frei programmierbare Schnittstellen aus um
beispielsweise Sensoren anzuschließen.

Mittlerweile gibt es mehrere Modelle:

\begin{itemize} 
\item Pi Zero 
\item Pi Zero W
\item Pi 1 Modell A
\item Pi 1 Modell A+
\item Pi 1 Modell B
\item Pi 1 Modell B+
\item Pi 2 Modell B
\item Pi 3 Modell B 
\end{itemize}


\section{Sprachen}

\subsection{Python}
\subsection{Java}

\subsection{SQL}

\subsection{HTML}

\subsection{PHP}

\subsection{Javascript}

\section{Sensoren}

Es werden folgende Sensoren verwendet:

\begin{itemize}
\item Feuersensor
\item
\end{itemize}

\section{WLAN}
\subsection{Standard}

Die Vernetzung erfolgt über ein Funknetz nach dem WLAN-Standard 802.11n. 

\subsection{Verschlüsselung}
WEP, WPA, WPA2
%%


