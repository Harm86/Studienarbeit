%Schreiben alle Zamme
\chapter{Einleitung}
\section{Zusammenfassung}
Im fünften und sechsten Semester an der Dualen Hochschule Baden-Württemberg ist es notwendig eine Studienarbeit durchzuführen. Die Studienarbeit soll ,wie auch die Projektarbeiten, als Vorbereitung auf die Bachelorarbeit dienen. 

Die Studienarbeit mit dem Thema "'Entwicklung eines Sensorknotens für IoT"' wurde mit Raspberry Pi's umgesetzt. Grundsätzlich soll Messwerten von einem oder mehreren Sensorknoten erfasst und angezeigt werden. Für die Speicherung, Verarbeitung und Anzeige der Messdaten sein soll ein zentraler Raspberry Pi verwendet werden. Daraus ergibt sich, dass es einen oder mehrere Sensorknoten gibt, die ihre Daten an eine Zentraleinheit senden. Diese speichert die Daten in einer Datenbank und über eine Webseite kann auf die Datenbank zugegriffen werden. Die Daten werden strukturiert und geordnet auf der Webseite aufbereitet. Durch Autorisierung und Authentifizierung können die Messdaten geschützt werden.

Ziel des Projekts ist es eine Umgebung zu erhalten, die erweiterbar, funktional und stabil ist. Es sollen nur Raspberry Pis verwendet werden, um zu zeigen, dass es für einen Sensorknoten nicht immer komplexe und teure Hardware braucht. Außerdem wird durch die konsequente Verwendung von Raspberry Pis die Komplexität im Rahmen gehalten. Diese würde größer werden, wenn viele unterschiedliche Komponenten zum Einsatz kommen würden. Da der Raspberry Pi nur digitale Signale und somit nur digitale Ein- und Ausgänge besitzt, ist Kompromiss notwendig. Manche Sensoren liefern jedoch analoge Werte, die erst durch einen \ac{A/D-Wandler} in ein digitales Signal gewandelt werden müssen. 

Für die Umsetzung der Webseite wurde eine Kombination Programmier und Darstellungsprachen verwendet. Durch die Verwendung von \ac{HTML}, JavaScript und \ac{PHP} ist eine dynamische Visualisierung der Messwerte möglich.

\section{Motivation}
Das Thema wurde von Herrn Professor H.-J. Haubner angeboten und hat uns direkt angesprochen. Die Anforderung von Herrn Haubner, dass der Sensorknoten auf Basis des Raspberry Pi's umgesetzt werden sollen fanden wir drei spannend, da es eine neue Herausforderung für uns darstellte. Die Auseinandersetzung mit den einzelnen Sensoren wurde ebenfalls als bewältigbare Herausforderung eingeschätzt. Interessant war vor allem auch, dass sowohl Datenbank-, Netzwerk- und Hardwarethemen zu bearbeiten waren. Während der Umsetzung hatten wir viele weitere Ideen.Jeder konnte sein fachliches Wissen in das Projekt einfließen lassen. 
