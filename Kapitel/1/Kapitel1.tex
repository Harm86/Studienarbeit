%Schreiben alle Zamme
\chapter{Einleitung} 
Im fünften und sechsten Semester muss an der Dualen Hochschule Baden-Württemberg eine Studienarbeit durchgeführt werden. Der Fokus soll im Gegensatz zu den Projektarbeiten der Praxisphasen mehr auf der Wissenschaftlichkeit liegen. Die Studienarbeit soll wie auch die Projektarbeiten als Vorbereitung auf die Bachelorarbeit dienen. \\
Das Thema Entwicklung eines Sensorknotens für IoT ist mit Raspberry Pis umgesetzt worden. Grundsätzlich soll ein oder mehrere Sensorknoten Messwerte anzeigen. In diesem Projekt wurde entschieden, dass ein zentraler Raspberry Pi für die Speicherung, Verarbeitung und Anzeige der Messdaten sein soll. Daraus ergibt sich, dass es einen oder mehrere Sensorknoten gibt, die ihre Daten an eine Zentraleinheit senden. Die Zentraleinheit speichert die Daten in einer Datenbank und über eine Webseite kann auf die Datenbank zugegriffen werden. Die Daten werden strukturiert und geordnet auf der Webseite aufbereitet. Eine Benutzerverwaltung ermöglicht es, dass nicht jeder auf die Messdaten zugreifen kann.\\
Ziel des Projekts ist es eine Umgebung zu erhalten, die erweiterbar, funktional und bla bl ist. Es sollen nur Raspberry Pis verwendet werden um zu zeigen, dass es für einen Sensorknoten nicht immer komplexe und teure Hardware braucht. Außerdem wird durch die konsequente Verwendung von Raspberry Pis die Komplexität im Rahmen gehalten, die größer würde wenn viele verschiedene Komponenten verwendet würden. An dieser Stelle ist ein Kompromiss notwendig, da der Raspberry Pi nur digitale Ein- und Ausgänge hat. Manche Sensoren liefern jedoch analoge Werte, die erst durch einen \ac{A/D-Wandler} in ein digitales Signal gewandelt werden müssen. Die Verwendung von PHP bei der Webseite ermöglicht es die Webseite auf wenige Programmiersprachen zu begrenzen. So ist durch die Verwendung von PHP lediglich \ac{HTML}, JavaScript und \ac{PHP} für die Webseite notwendig, was wieder die Komplexität reduziert.

\section{Motivation}
Das Thema wurde von Herrn Professor H.-J. Haubner angeboten und hat uns direkt angesprochen. Die Anforderung von Herrn Haubner, dass der Sensorknoten auf Basis des Raspberry Pis umgesetzt werden soll fanden wir drei spannend da bereits in vorherigen Projekten mit Raspberry Pis gearbeitet wurde. Die Auseinandersetzung mit den einzelnen Sensoren wurde ebenfalls als bewältigbare Herausforderung eingeschätzt. Interessant war vor allem auch, dass sowohl Datenbank-, Netzwerk- und Hardwarethemen zu bearbeiten waren. Während der Bearbeitung hatten wir noch viele weitere Ideen. Diese Ideen konnten jedoch aus Zeitgründen nicht implementiert werden. Die unterschiedlichen fachlichen Hintergründe der drei Projektteilnehmer haben sehr gut zusammengespielt und jeder konnte sich einbringen und seinen Teil zum Gesamtprojekt beitragen.
