%%%%%%%%%%%%%%%%%%%%%%%%%%%%%%%%% Packages/Dokumentart %%%%%%%%%%%%%%%%%%%%%%%%%%%%%%%%%%%%%%%%%%%%%%%%%%%%%%%%%%%%%%%%%%%%%%%%%%%%%%%%%%%%%%%%%%

\documentclass[ a4paper,	% Papierart
	% 10pt,		
	%	11pt,									% Schriftgröße
		12pt,									
		pdftex,								% PDF Umwandlung
	%	twoside								% 2-seitig
		] {report}						% Dokumenttyp Bericht

\usepackage[ngerman]{babel}			% Deutsches Sprachpaket/Silbentrennung etc.
\usepackage[utf8]{inputenc}			% verwendeter Codec (für Umlaute)
\usepackage{graphicx}						% für Bilder
\usepackage{cite}	%					% für bestimmte Zitierfunktionen
\usepackage[footnote]{acronym}	% für Abkürzungen in Fußnote
\usepackage{caption}						% Um Bildunterschriften zu konfigurieren (mit, ohne erscheinen im Abkürzungsverzeichnis)
\usepackage{setspace}						% für Zeilenabstand
\usepackage{url}   							% Package zur Darstellung von URL's
\usepackage{eurosym}  					% Package zur Darstellung von €-Zeichen
\usepackage{booktabs}
\usepackage[bookmarksopen,
						bookmarksnumbered,
						plainpages=false,		%vermeidet diverse Warnungen
						pdfpagelabels=true,	%ebenso
						]{hyperref}
\usepackage{etoolbox}  		% http://dante.ctan.org/tex-archive/help/Catalogue/entries/etoolbox.html
\usepackage{csquotes}
%%\usepackage[
%%	hyperref=true,          % Klickbare Referenzen in der PDF-Datei
%%  backref=true,           % In der Literaturref. die Seiten angeben, wo ein
  % \cite dazu steht
%%  bibencoding=inputenc,   % s. inputenc-Paket
%%  backend = biber,
%%  style = alphabetic,			% [Aut98]
	%style=authoryear-comp,	% Autor Jahr
	%style=numeric-comp,		% [1]
%%	sorting=nty]{biblatex}  %
	% http://dante.ctan.org/tex-archive/help/Catalogue/entries/biblatex.html

%%\addbibresource{Literaturverzeichnis.bib}
%%
\newcommand{\tabitem}{~~\llap{\textbullet}~~}			% itemize in einer \tabular Umgebung

\onehalfspacing									% ab hier Zeilenabstand 1,5

\usepackage[a4paper, left=2.5cm, right=2.5cm,
top=2.5cm, bottom=2.5cm]{geometry}

\setlength{\headheight}{1.1\baselineskip}

%%%%%%%%%%%%%%%%%%%%%%%%%%%%%%%%%%%%%%%%%%%%%% Kopfzeile %%%%%%%%%%%%%%%%%%%%%%%%%%%%%%%%%%%%%%%%%%%%%%%%%%%%%%%%%%%%%%%%%%%%%%%%%%%%%%%%%%%%

\usepackage[automark]{scrpage2}				% Package für Kopfzeile	
\pagestyle{scrheadings} 							% Pagestyle
\automark[section]{chapter}						% für Anzeigen des Unterkapitels (Standard Überkapitel)
\clearscrheadfoot 
\ihead{\headmark}											% ihead links, ohead rechts, chead mitte
\ohead{Vorname Nachname}											
\setheadsepline{0.4pt}								% Linie unter Kopfzeile

%%%%%%%%%%%%%%%%%%%%%%%%%%%%%%%%%%%%%%%%%%%%%% Abkürzungen %%%%%%%%%%%%%%%%%%%%%%%%%%%%%%%%%%%%%%%%%%%%%%%%%%%%%%%%%%%%%%%%%%%%%%%%%%%%%%%%%%

% Persönlich
\newcommand{\Autor}{Jan Mannherz, Alexander Sinicyn, Harm-Christian Schweizer}
\newcommand{\MatrikelNummer}{1899163, 9617383, 2161207}
\newcommand{\Kursbezeichnung}{Tinf14B3}
%Firma
%\newcommand{\FirmenName}{EDEKA Handelsgesellschaft Südwest mbH}
%\newcommand{\FirmenNameKurz}{EDEKA Südwest}							
%\newcommand{\FirmenStadt}{77656 Offenburg}							
%\newcommand{\FirmenLogoDeckblatt}{\includegraphics[width=1.8cm]{Bilder/edekaneu.png}}	
\newcommand{\dhLogo}{\includegraphics[width=4cm]{Bilder/dhbw-logo.png}}
%Betreuer
%\newcommand{\BetreuerFirma}{Betreuer}
\newcommand{\BetreuerDHBW}{Herr Prof. Dr. Haubner}

%%%%%%%%%%%%%%%%%%%%%%%%%%%%%%%%%%%%%%%%%%%%%% Art der Arbeit %%%%%%%%%%%%%%%%%%%%%%%%%%%%%%%%%%%%%%%%%%%%%%%%%%%%%%%%%%%%%%%%%%%%%%%%%%%%%%

%\newcommand{\Was}{Praxisbericht}
%\newcommand{\WasErklaerung}{den vorliegenden \Was}
%\newcommand{\Was}{Projektarbeit}
%\newcommand{\WasErklaerung}{die vorliegende \Was}
\newcommand{\Was}{Studienarbeit}
\newcommand{\WasErklaerung}{die vorliegende \Was}
%\newcommand{\Was}{Bachelorarbeit}
%\newcommand{\WasErklaerung}{die vorliegende \Was}

%%%%%%%%%%%%%%%%%%%%%%%%%%%%%%%%%%%%%%%%%%%%%% Deckblatt Infos %%%%%%%%%%%%%%%%%%%%%%%%%%%%%%%%%%%%%%%%%%%%%%%%%%%%%%%%%%%%%%%%%%%%%%%%%%%%%%%

\newcommand{\Titel}{Entwicklung eines Sensorknotens für IoT}
%\newcommand{\Dauer}{12 Wochen}
\newcommand{\Abschluss}{Bachelor of Engineering}
\newcommand{\Studiengang}{Informationstechnik}
\newcommand{\AbgabeDatum}{29.05.2017}
%
\begin{document}

%%%%%%%%%%%%%%%%%%%%%%%%%%%%%%%%%%%%%%%%%%%%%%%%%%%% Titelseite %%%%%%%%%%%%%%%%%%%%%%%%%%%%%%%%%%%%%%%%%%%%%%%%%%%%%%%%%%%%%%%%%%%%%%%%%%%%%%

\hypersetup{pageanchor=false}						% vermeidet falsche Referenzierung durch gleiche Seitenanzahlen

\begin{singlespace}											% Zeilenabstand für Titelseite verringern
\begin{titlepage}
\begin{center}													% Referenzpunkt Seitenmitte
\vspace*{-2cm}													% 2 cm nach Links Platz lassen
\hfill\includegraphics[width=4cm]{Bilder/dhbw-logo}\\[2cm]  % Firmenlogo
% platzieren

{\Huge \Titel}\\[2cm]										% \Huge, \Large, \large sind versch. Schriftgrößen \bfseries ist Fett gedruckt
{\Huge\scshape \Was}\\[2cm]							% [] Inhalt ist Abstand zur nächsten Zeile
{\large für die Prüfung zum}\\[0.5cm]
{\Large \Abschluss}\\[0.5cm]
{\large des Studienganges \Studiengang}\\[0.5cm]
{\large an der}\\[0.5cm]
{\large Dualen Hochschule Baden-Württemberg Karlsruhe}\\[0.5cm]
{\large von}\\[0.5cm]
{\large\bfseries \Autor}\\[1cm]
{\large Abgabedatum \AbgabeDatum}
\vfill																	% ermöglicht es unterhalb der Seitengrenze zu schreiben
\end{center}														% Referenzpunkt Mitte beenden

\begin{tabular}{l@{\hspace{2cm}}l}			% Beginn Tabelle mit einer Spalte nach der 2cm Platz gelassen wird bevor die nächste beginnt
%Bearbeitungszeitraum           	& \Dauer 				\\
Matrikelnummer	             	& \MatrikelNummer		\\
Kurs			         		& \Kursbezeichnung		\\
Gutachter der Studienakademie	& \BetreuerDHBW			\\
\end{tabular}														% Tabelle abschließen
\end{titlepage}													% Titelseite abschließen
\end{singlespace}												% ab hier wieder 1.5 fach Zeilenabstand

%%%%%%%%%%%%%%%%%%%%%%%%%%%%%%%%%%%%%%%%%%%%%%% Erklärung & Sperrvermerk %%%%%%%%%%%%%%%%%%%%%%%%%%%%%%%%%%%%%%%%%%%%%%%%%%%%%%%%%%%

%\setlength{\parindent}{0em}
%%%%%%%%%%%%%%%%%%%%%%%%%%%%%%%%%%%%%% Erklaerung %%%%%%%%%%%%%%%%%%%%%%%%%%%%%

\newpage
\thispagestyle{empty}

\begin{center}
\Large\bfseries Erkl\"arung
\end{center}

\noindent
Gem\"a\ss{} \S~5 (3) der "`Studien- und Prüfungsordnung f\"ur den Studienbereich
Technik"' vom 22.09.2011.

\medskip
\noindent
Wir haben \WasErklaerung\ selbstst\"andig verfasst und
keine anderen als die angegebenen Quellen und Hilfsmittel verwendet.

\vspace{3cm}
\noindent
\underline{\hspace{4cm}}\hfill\underline{\hspace{7.25cm}}\\
Ort,~~~~~Datum\hfill Unterschrift Jan Mannherz\hspace{1.25cm}

\vspace{3cm}
\noindent
\underline{\hspace{4cm}}\hfill\underline{\hspace{7.25cm}}\\
Ort,~~~~~Datum\hfill Unterschrift Alexander Sinicyn\hspace{0.80cm}

\vspace{3cm}
\noindent
\underline{\hspace{4cm}}\hfill\underline{\hspace{7.25cm}}\\
Ort,~~~~~Datum\hfill Unterschrift Harm-Christian Schweizer

%\input{Anhang/Sperrvermerk.tex}

%%%%%%%%%%%%%%%%%%%%%%%%%%%%%%%%%%%%%%%%%%%%%%%% Kurzzusammenfassung %%%%%%%%%%%%%%%%%%%%%%%%%%%%%%%%%%%%%%%%%%%%%%%%%%%%%%%%%%%%%%%%%%%%%%%%%%%%%%%%%%%%%%

\begin{abstract}
\begin{onehalfspace}

Kurze Zusammenfassung (Abstract)

\end{onehalfspace}
\end{abstract} 

%%%%%%%%%%%%%%%%%%%%%%%%%%%%%%%%%%%%%%%%%%%%% Verzeichnisse %%%%%%%%%%%%%%%%%%%%%%%%%%%%%%%%%%%%%%%%%%%%%%%%%%%%%%%%%%%%%%%%%%%%%%%%%%%%%%%%%%%%%%%%%%%

\cfoot[\pagemark]{\pagemark}							% Mitte Fußzeile Seitenzahl (selbe wie mit head (i,o,c))
\pagenumbering{Roman}         						% Seitenzahlen für Verzeichnisse auf römisch

\begin{singlespace}												% Zeilenabstand für Verzeichnisse 1	
\tableofcontents 			 										% Inhaltsverzeichnis
\listoffigures	 			 										% Abbildungsverzeichnis
\listoftables				 	 										% Tabellenverzeichnis
\end{singlespace}	 		 										% Zeilenabstand wieder ausstellen
%\listofequations			 										% Formelverzeichnisir
%\listoflistings 			 										% Listenverzeichnis
\chapter*{Abkürzungsverzeichnis}		  % das * verhindert das das Kapitel eine Nummerierung erhält
\begin{singlespace}				  					% Einfacher Zeilenabstand wegen Platz

\begin{acronym}[wwwwwwwwwwwwwwwww]    % Inhalt der eckigen Klammer ist der Abstand von der Abkürzung zur Erklärung/ Beginn der Acro Umgebung
	\setlength{\itemsep}{-\parsep}		  % Veringert den Abstand untereinander
	\acro{A/D-Wandler}{Analog/Digital-Wandler}
	\acro{AES}{Advanced Encryption Standard}
	\acro{AJAX}{Asynchronous JavaScript and XML}
	\acro{CCMP}{Counter Mode with Cipher Block Chaining Message Authentication Code Protocol}
	\acro{CRC}{Cyclic Redundancy Check}
	\acro{CSS}{Cascading Style Sheets}
	\acro{DHCP}{Dynamic Host Configuration Protocol}
	\acro{GPIO}{General Purpose Input Output}
	\acro{HTML}{Hypertext Markup Language}
	\acro{IEEE}{Institute of Electrical and Electronics Engineers}
	\acro{IV}{Initialisierungsvektor}
	\acro{JSON}{JavaScript Object Notation}
	\acro{LLC}{Logical Link Control}
	\acro{MIMO}{Multiple Input Multiple Output}
	\acro{OOP}{objektorientierte Programmierung}
	\acro{PHP}{PHP: Hypertext Preprocessor}
	\acro{PMK}{Pairwise Master Key}
	\acro{Poti}{Potentiometer}
	\acro{PPE}{Per-Packet-Encryption}
	\acro{PSK}{Pre-Shared-Key}
	\acro{QoS}{Quality of Service}
	\acro{SSID}{Service Set Identifier}
	\acro{SQL}{Structured Query Language}
	\acro{TKIP}{Temporal Key Integrity Protocol}
	\acro{WEP}{Wired Equivalent Privacy}
	\acro{WPA}{Wi-Fi Protected Access}
	\acro{WPA2}{Wi-Fi Protected Access 2}
\end{acronym}					  							% Ende der Acro Umgebung

\end{singlespace}
 % Abkuerzungsverzeichnis
\newpage

\hypersetup{pageanchor=true} 							% ab hier werden wieder Anker gesetzt und damit richtige Referenzen gebildet

%%%%%%%%%%%%%%%%%%%%%%%%%%%%%%%%%%%%%%%%%%%%%%%%%%% Kapitel einbinden %%%%%%%%%%%%%%%%%%%%%%%%%%%%%%%%%%%%%%%%%%%%%%%%%%%%%%%%%%%%%%%%%%%%%%%%%

\pagenumbering{arabic}                  % Ab hier Seitenzahl auf normal

%Schreiben alle Zamme
\chapter{Einleitung}
\section{Zusammenfassung}
Im fünften und sechsten Semester an der Dualen Hochschule Baden-Württemberg ist es notwendig eine Studienarbeit durchzuführen. Die Studienarbeit soll ,wie auch die Projektarbeiten, als Vorbereitung auf die Bachelorarbeit dienen. 

Die Studienarbeit mit dem Thema "'Entwicklung eines Sensorknotens für IoT"' wurde mit Raspberry Pi's umgesetzt. Grundsätzlich soll Messwerten von einem oder mehreren Sensorknoten erfasst und angezeigt werden. Für die Speicherung, Verarbeitung und Anzeige der Messdaten sein soll ein zentraler Raspberry Pi verwendet werden. Daraus ergibt sich, dass es einen oder mehrere Sensorknoten gibt, die ihre Daten an eine Zentraleinheit senden. Diese speichert die Daten in einer Datenbank und über eine Webseite kann auf die Datenbank zugegriffen werden. Die Daten werden strukturiert und geordnet auf der Webseite aufbereitet. Durch Autorisierung und Authentifizierung können die Messdaten geschützt werden.

Ziel des Projekts ist es eine Umgebung zu erhalten, die erweiterbar, funktional und stabil ist. Es sollen nur Raspberry Pis verwendet werden, um zu zeigen, dass es für einen Sensorknoten nicht immer komplexe und teure Hardware braucht. Außerdem wird durch die konsequente Verwendung von Raspberry Pis die Komplexität im Rahmen gehalten. Diese würde größer werden, wenn viele unterschiedliche Komponenten zum Einsatz kommen würden. Da der Raspberry Pi nur digitale Signale und somit nur digitale Ein- und Ausgänge besitzt, ist Kompromiss notwendig. Manche Sensoren liefern jedoch analoge Werte, die erst durch einen \ac{A/D-Wandler} in ein digitales Signal gewandelt werden müssen. 

Für die Umsetzung der  Webseite wurde eine Kombination Programmier und Darstellungsprachen verwendet. Durch die Verwendung von \ac{HTML}, JavaScript und \ac{PHP} ist eine dynamische Visualisierung der Messwerte möglich.

\section{Motivation}
Das Thema wurde von Herrn Professor H.-J. Haubner angeboten und hat uns direkt angesprochen. Die Anforderung von Herrn Haubner, dass der Sensorknoten auf Basis des Raspberry Pi's umgesetzt werden sollen fanden wir drei spannend, da es eine neue Herausforderung für uns darstellte. Die Auseinandersetzung mit den einzelnen Sensoren wurde ebenfalls als bewältigbare Herausforderung eingeschätzt. Interessant war vor allem auch, dass sowohl Datenbank-, Netzwerk- und Hardwarethemen zu bearbeiten waren. Während der Umsetzung hatten wir viele weitere Ideen.Jeder konnte sein fachliches Wissen in das Projekt einfließen lassen. 

\newpage

\chapter{Grundlagen}

\section{Programmiersprachen}

\subsection{Python vs. Java}

\subsection{MySQL}

\section{Raspberry Pi}
Der Raspberry Pi wurde von der britischen Raspberry Pi Foundation entworfen um
jungen Menschen den Erwerb von Programmier- und Hardwarekenntnissen zu
ermöglichen. Er ist ein Einplatinencomputer und für wenig Geld verfügbar. Der
Raspberry Pi zeichnet sich durch frei programmierbare Schnittstellen aus um
beispielsweise Sensoren anzuschließen.

 Mittlerweile gibt es mehrere Modelle:

\begin{itemize} 
\item Pi Zero 
\item Pi Zero W
\item Pi 1 Modell A
\item Pi 1 Modell A+
\item Pi 1 Modell B
\item Pi 1 Modell B+
\item Pi 2 Modell B
\item Pi 3 Modell B 
\end{itemize}


\section{Konfiguration}

Die Sensoren werden an einen Raspberry Pi angeschlossen und melden die
gemessenen Werte an einen zentralen Raspberry Pi. Der zentrale Raspberry Pi legt
die gemeldeten Daten in einer Datenbank ab. Eine Website greift auf die
Datenbank zu und stellt die Daten dar. 

\section{Sensoren}

Es werden folgende Sensoren verwendet:

\begin{itemize}
\item Feuersensor
\item
\end{itemize}

\section{Vernetzung}

Die Vernetzung erfolgt über ein Funknetz nach dem WLAN-Standard 802.11n. 

\section{Verschlüsselung}

Das WLAN ist mit WPA2 verschlüsselt. WPA2 gilt aktuell als sicher, was nicht für
die Alternativen WEP oder WPA gilt. Zur Authentifizierung wird ein
Pre-Shared-Key (PSK) verwendet.

\newpage

\chapter{Kapitel 3}

\section{Überschrift}

%%%%%%%%%%%%%%%%%%%%%%%%%%%%%%%%%%%%%%%%%%%%%%%%%%%%%%%%%%%%%%%%%%%%%%%%%%%%%%%%%%%%%%%%%%%%%%%%%%%%%%%%%%%%%%%%%%%%%%%%%%%%

\section{Überschrfit mit Label}\label{sec:JMeter}


\newpage

\chapter{Kapitel 4}

\section{Überschrift}

\subsection{Unter-Überschrift}

\subsubsection{Unter-Unter-Überschrift}

\newpage

\chapter{Fazit}

%\section{Problemstellungen}
%Bsp: Mesh, Sensoren falsch verlötet

\section{Ausblick}

\section{Fazit}
%%%
\newpage

%%%%%%%%%%%%%%%%%%%%%%%%%%%%%%%%%%%%%% URL wegen Sonderzeichen Vordefinieren %%%%%%%%%%%%%%%%%%%%%%%%%%%%%%%%%%%%%%%%%%%%%%%%%%%%%%%%%%%%%%%%%%%%%%%

%\urldef{\newUrl}\url{http://www.htl-hl.ac.at/el/projekte/2011/DA_Spielekonsole/Sonstiges/GanttProject%20Tutorial.pdf}

%%%%%%%%%%%%%%%%%%%%%%%%%%%%%%%%%% Literaturverzeichnis %%%%%%%%%%%%%%%%%%%%%%%%%%%%%%%%%%%%%%%%%%%%%%%%%%%%%%%%%%%%%%%%%%%%%%%%%%%%%%%%%%%%%%%%%%

\addcontentsline{toc}{chapter}{Literaturverzeichnis}
%\bibliographystyle{gerabbrv}					% Zitierformat (deutsch für Großschreibung) [Häufigkeit] im Text
%\bibliography{Literaturverzeichnis}		% Pfad des Literaturverzeichnisses.bib

\printbibliography
%Neues Quellenverzeichnis
\begin{thebibliography}{999}
	\bibitem{ALLNetSensoren} ALLNET: 4duino Sensor Kit Handbuch.\\
	\url{http://www.allnet.de/fileadmin/transfer/products/111861.pdf} besucht am	31.10.2016
\end{thebibliography}
\newpage
\thispagestyle{plain}
\chapter*{Anhang}
\addcontentsline{toc}{chapter}{Anhang}
\markboth{Anhang}{Anhang}
\begin{landscape}
	\begin{figure}[htb]
		\includegraphics[width=\linewidth, height=.9\textheight]{Bilder/Anhang/uebersicht.jpg}
		\caption[]{Übersichtseite}
		\label{uebersichtseite}
	\end{figure}
\end{landscape}
\begin{landscape}
	\begin{figure}[htb]
		\includegraphics[width=\linewidth, height=.9\textheight]{Bilder/Anhang/statistik.jpg}
		\caption[]{Statistikseite}
		\label{statistikseite}
	\end{figure}
\end{landscape}
\begin{landscape}
	\begin{figure}[htb]
		\includegraphics[width=\linewidth, height=.9\textheight]{Bilder/Anhang/webcam.jpg}
		\caption[]{Webkameraseite}
		\label{webkameraseite}
	\end{figure}
\end{landscape}
\begin{landscape}
	\begin{figure}[htb]
		\includegraphics[width=\linewidth, height=.9\textheight]{Bilder/Anhang/impressum.jpg}
		\caption[]{Impressumseite}
		\label{impressum}
	\end{figure}
\end{landscape}
							% Anhang Einbinden

\end{document}